\documentclass[twocolumn,,10pt]{IEEEtran}
\let\labelindent\relax
\usepackage{enumitem}
\input{preamble.tex} 
% \usepackage{geometry}
% \geometry{a4paper, left=.6in,right=.6in,top=.8in,bottom=0.7in}
\usepackage{textcomp}
\usepackage{colortbl}
\usepackage{subfigure}
\usepackage{array}
\usepackage{courier}
\usepackage{wrapfig}
\usepackage{pifont}
\usetikzlibrary{chains,backgrounds}
\usetikzlibrary{intersections}
\usetikzlibrary{pgfplots.groupplots}
\usepgfplotslibrary{fillbetween}
\usepackage{pgfplotstable}
\usepackage[super]{cite} 
\usepackage{setspace} 
\makeatletter \renewcommand{\@citess}[1]{\raisebox{1pt}{\textsuperscript{[#1]}}} \makeatother
\usepackage{xstring}
\usepackage{xspace}
\usepackage{flushend}
\makeatletter
\renewcommand\section{\@startsection {section}{1}{\z@}%
  {-2ex \@plus -1ex \@minus -.2ex}%
  {1ex \@plus.1ex}%
  {\large\bfseries\scshape}}
\renewcommand\subsection{\@startsection {section}{1}{\z@}%
  {-1ex \@plus -.25ex \@minus -.2ex}%
  {0.1ex \@plus.0ex}%
  {\fontsize{11}{10}\selectfont\bfseries\sffamily\color{DodgerBlue4}}}
\renewcommand\subsubsection{\@startsection {section}{1}{\z@}%
  {0ex \@plus -.5ex \@minus -.2ex}%
  {0.0ex \@plus.5ex}%
  {\fontsize{9}{9}\selectfont\bfseries\sffamily\color{Red4}}}
\renewcommand\paragraph{\@startsection {section}{1}{\z@}%
  {-1.5ex \@plus -.5ex \@minus -.2ex}%
  {0.0ex \@plus.5ex}%
  {\fontsize{9}{9}\selectfont\itshape\sffamily\color{teal!50!black}}}


\makeatother
\makeatletter
\pgfdeclareradialshading[tikz@ball]{ball}{\pgfqpoint{-10bp}{10bp}}{%
  color(0bp)=(tikz@ball!30!white);
  color(9bp)=(tikz@ball!75!white);
  color(18bp)=(tikz@ball!90!black);
  color(25bp)=(tikz@ball!70!black);
  color(50bp)=(black)}
\makeatother
\newcommand{\tball}[1][CadetBlue4]{${\color{#1}\Large\boldsymbol{\blacksquare}}$}
\renewcommand{\baselinestretch}{.95}
\renewcommand{\captionN}[1]{\caption{\color{CadetBlue4!80!black} \sffamily \fontsize{8}{9}\selectfont #1  }}
\tikzexternaldisable 
\parskip=7pt
\parindent=0pt
\newcommand{\Mark}[1]{\textsuperscript{#1}}
% \lhead{\sf\footnotesize \color{DodgerBlue4!70!black}\today}
\pagestyle{fancy}
\def\COLA{black}
% ###################################
\cfoot{\bf\sffamily \scriptsize \color{Maroon!50} \disclosure }
\cfoot{}
% \lhead{\sffamily \scriptsize \color{DodgerBlue4!70!black} Algorithms for Threat Detection NSF 17-510}
\rhead{\scriptsize\bf\sffamily\thepage}
\newcommand{\partxt}{\bf\sffamily\itshape}
% ############################################################
\newif\iftikzX
\tikzXtrue
\tikzXfalse
\def\jobnameX{atd}
\newcommand{\SPX}[1][50pt]{\vspace{#1}}
% ############################################################
\newcommand{\incomplete}{\colorbox{Red1!80}{\textbf{\footnotesize\color{white}(incomplete section)}}}
\def\FWN{\textbf{\small FWN}\xspace}
% ############################################################
\begin{document} 
% 
% \chead{\bf\sffamily \footnotesize \color{DodgerBlue4!90!black} I. Chattopadhyay, University of Chicago}
% 
\twocolumn[
\xtitaut{\bf\sffamily \Large \color{black!90!DodgerBlue1}  \fontsize{16}{18}\selectfont 
  {Prediction \& Intervention of   Criminal Infractions in Urban Environments\\
    \vskip .1em
    \small \rm Ishanu Chattopadhyay (ishanu@uchicago.edu)
    % \hrule 
  } 
}{}]{}
\vspace{-10pt}   

\begin{abstract}
  Prediction of social phenomena.
\end{abstract}

% ++++++++++++++++++++++++++++++++++++++++++++++++++++


%###########################################################
%###########################################################
\begin{figure*}[!ht]
  \tikzexternalenable

  \centering 

  \input{Figures/figpred1}

  %\vspace{-15pt}

  \captionN{We see a strong associationn with spatial distribution of socio-economic indicators and the distribution of predicted enforcement response to perturbations in crime rates. With a 10\% increase in each of violent and non-violent crime rates, we  have approximatey a $30\%$ decrease in arrests when averaged over the city. However, the spatial distribution of locations which experience a positive vs a negative change in the rate of arrests reveals a strong preference for favorable socio-economic indicators for the former. Thus, if neighborhoods are doing better socio-economically, increasing crime seems to predict increased arrests. A strong converse trend is observed in our predictions for neighborhoods doing worse. }\label{fig0}
\end{figure*}
%###########################################################
%###########################################################
%###########################################################


note:
fig0 is aboutr problem
fig1 is abpout prediction

fig2 is about diseses indicators

fig4 is about methodology

\end{document}
















